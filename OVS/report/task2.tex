\section*{Task 2}
\addcontentsline{toc}{section}{Task 2}
The stack of integers uses a singly linked list based on the following data structure:
\begin{lstlisting}
// forward declaration of struct and typedef
typedef struct node Node;

// Linked list data structure
struct node {
    int data;
    Node* next;
};
\end{lstlisting}

Popping from an empty stack has in general undefined behavior.\footnote{See \url{http://stackoverflow.com/questions/7390126}} Thus we chose to include a function to check if a stack is empty so that the programmer can check the stack before popping. If a pop happens from an empty stack, the program displays an error message and aborts.

\begin{lstlisting}
void push(Node** headRef, int newData) {
    // allocate node
    Node* newNode = (Node*) my_malloc(sizeof(Node));
    if (!newNode) return;
            
    newNode->data = newData;
    newNode->next = (*headRef);
    (*headRef) = newNode;
}

int pop(Node** headRef) {
    Node* head = *headRef;
    if(!head) { fputs("Error: pop from empty stack\n", stderr);  abort();  }
    int result = head->data; // pull out the data before the node is deleted
    *headRef = head->next;
    my_free(head); // free the head node
    return result;
}

int is_empty(Node* head) {
    return head == NULL;
}
\end{lstlisting}

Our test harness consists of a \texttt{main} which simply initializes a empty stack, pushes a large number of integers (value $i$ in the $i$th iteration) and then popping until the stack is empty and checking the bounds of the popped elements. We first tested the program using some \texttt{printf} statements for 1024 elements by running it manually.
\begin{lstlisting}
int main() {
    Node* stack = NULL; //initialize stack
    int i,popped;
    for (i=0; i < NB_ELEMENTS; ++i) { push(&stack, i);  }
    while( stack ) {  popped = pop(&stack); assert( popped >= 0 && popped < NB_ELEMENTS); }
    assert(counter==0);
    return 0;
}
\end{lstlisting}
To check the program for the absence for \texttt{NULL} pointer dereferences, buffer overflows and pointer-safety errors using \texttt{cbmc}, we then again instrumented the \texttt{malloc} function (as in Task~1) and ran the program with the \verb|--pointer-check| and \verb|--bounds-check| options (first separately, then together). We also added a counter for keeping track of the number of \texttt{malloc}s minus \texttt{free}s to check for memory leaks by putting a \texttt{assert} statement at the end that this number is zero.
\begin{lstlisting}
#ifdef __VERIFY
int counter = 0;
_Bool nondet_bool ();
void *my_malloc (size_t s)
{ 
  if (nondet_bool()) { return (NULL); }
  
  counter++;
  return (malloc (s)); 
}
 
void my_free(void* arg)
{
  free(arg);
  counter--;
}
(...)
#endif
\end{lstlisting}

The while loop in the main function has no (useful) run-time bound. Thus,
the \verb|--unwind| parameter has to be used---using (trivial) loop-bounds, that is, the number of elements---in order to prevent infinite unwinding.

When we tried to verify the program using the test harness for many elements ($n > 1000$), the verification step took too long to complete. This is a manifestation of the state explosion problem, caused by the fact that when creating an element, it can have either a successful or failed \texttt{malloc} operation, which results in $2^n$ possible paths. A couple of computation times are shown: they confirm that \texttt{cbmc} does not scale well for a large number of elements in this program.
\begin{center}
\begin{tabular}{c | c c c c c c}
$n$ & 4 & 8 & 12 & 16 & 20 & 24 \\ \hline
time (s) & 0.1 & 0.9 & 15 & 199 & 1828 & 11728 \\
\end{tabular}
\end{center}

Combinations with \verb|--bounds-check| and/or \verb|--pointer-check| options also do not pose extra problems, e.g., for $n=8$, following verification succeeds in $1.6$\,s.
\begin{Verbatim}[formatcom=\color{red}]
cbmc stack_int.c -D __VERIFY --unwind 9 --pointer-check --bounds-check
\end{Verbatim}
When using smaller loop bounds, \texttt{cbmc} obviously reports unwinding assertion errors.

