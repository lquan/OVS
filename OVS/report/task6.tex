\section*{Task 6}
\addcontentsline{toc}{section}{Task 6}
%As \cite{Shankar01detectingformat} mentions, static code analysis tools are not well-suited for this task.
It is not clear how \texttt{cbmc} should be used to detect format string vulnerabilities~\cite{lhee,Younan04codeinjection}.  However, \emph{possible} format string vulnerabilities are usually very easy to detect: 
\begin{itemize}
 \item manually, which can be done only on short programs;
 \item using a \texttt{grep}-like approach (\texttt{PScan}) or more advanced static analysis tools such as \texttt{flawfinder}, \texttt{RATS} and \texttt{ITS4};
 \item using compiler flags (\texttt{gcc} using e.g.~\texttt{-Wformat} which was even a default parameter on our system);
 \item using existing compiler extensions such as FormatGuard~\cite{formatguard} which a.o.~perform argument counting.
\end{itemize}

E.g., taking the most basic format string vulnerability example in~\cite{lhee}
\begin{lstlisting}
void vulnfunc (char *user)
{
    printf(user);
}
\end{lstlisting}
and compiling with \texttt{gcc} gives following warning:
\begin{Verbatim}[fontsize=\footnotesize]
string_vul.c: In function ‘vulnfunc’:
string_vul.c:7: warning: format not a string literal and no format arguments
\end{Verbatim}